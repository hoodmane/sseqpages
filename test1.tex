\documentclass{article}
\usepackage[landscape,margin=0cm,top=2cm]{geometry}
\usepackage{tikz}
\makeatletter
\usepackage{sseqpages}
\usepackage{amsmath}
\usepackage{amssymb}
\usepgflibrary{shapes.misc}
\makeatletter
%\show\sseq@ifundefcoord@
%\traceon
%\edef\test{\sseq@testpt 1.south.\@nil}
%\edef\testa{\sseq@testpt 1.9.\@nil}
%\show\test
%\show\testa
%\def\sseq@thename{thename}
%\def\coord{(2,2)}
%\tracingall
%\sseq@setcoordprop{propname}(2,2){hi}
%\edef\test{\sseq@getcoordprop{propname}\coord}
%\show\test

\def\row#1{\foreach \x in {1,...,35}{\place(\x,#1)}}
\def\twoptrow#1{\foreach \x in {1,...,35}{\place(\x,#1)\place(\x,#1)}}
%\sseqset{every differential/.style={->},every class/.style=blue}

\begin{document}

%\sseq@errorcheckingfalse
%\sseq@setstdoffsets12{-0.13}{+0.13}
%\sseq@setstdoffsets22{+0.13}{-0.13}
\tracingall
\begin{sseqdata}[execute at begin node=$,execute at end node=$,differentials={->,shorten >=5pt,blue},classes={draw,fill=white},scale=0.9]{mysseq}
\degree{-#1}{#1-1}
\cyclestyles{circle,inner sep=0.3ex}{green,fill=white,inner sep=0.3ex}

\xrange{1}{25}
\yrange{0}{19}%

\draw[background,step=1cm,gray,very thin] (1,0) grid (25,19);
\structline(1,7)(1,8)

\sseqset{every node/.style=blue}
\place[alias=hi,xshift=6,text=black,text height=2em,"\mathbb{Z}"](1,6)
\place[needs tikz,alias=there,text=black,text height=2em,"\mathbb{Z}"](4,6)
\place[needs tikz, text=black, "\mathbb{Z}"](1,5)
\place[transform shape,text opacity=0.2,"\mathbb{Z}"](2,6)
\place[transform shape,needs tikz, "\mathbb{Z}"](2,5)
\foreach \x in {1,3,...,35} {\place["\mathbb{Z}",double,rounded corners=1pt](\x,0)}
\draw (hi) -- (there);
%\draw (0,0)--(15,20);
%\draw[standard parse,scale=0.5] (10,1)--(10,10);
%\draw[standard parse] (7,0)--(3,6);

\begin{scope}[background, xshift=1cm,scale=0.5]
%\sseq@g@addtosavedpaths{\show\error}
\draw (0,0)--(30,30);
\end{scope}


\sseqset{every node/.style=red}

\place[chamfered rectangle,double,"\mathbb{Z}"](3,6)
\row{1}
\row{2}
\row{3}

\cyclestyles{draw,circle,fill,inner sep=0.3ex}
            {draw,green,rectangle,fill,inner sep=0.3ex}

%\begin{scope}[opacity=0.6]
\foreach \y in {7,15,23}{
    \row{\y}
    \row{\y+1}
    \twoptrow{\y+2}
    \row{\y+3}
    \row{\y+4}
}
%\end{scope}

\place(1,12)
\draw(1,2)--(1,3);
\node at (1,13) {a!};
\d[draw=none]2(5,1)

\d[bend left=30]2(3,0)

%\d2(4,0)
\structline(10,7)(10,8)
\begin{scope}[xshift=2]
\place(1,12)
\draw(1,2)--(1,3);
\node at (1,13) {a!};
\d["hi"]2(3,0)
\structline(10,7)(10,8)
\end{scope}




%\@xp\show\csname sseq@feature0\endcsname
%\@xp\show\csname sseq@featureinfty0\endcsname
%\@xp\show\csname sseq@savednodes@mysseqinfty\endcsname
%\@xp\show\csname sseq@transferfeature0\endcsname
%\@xp\show\csname sseq@clearinfty\sseq@thename(9,2)\endcsname
%\@xp\show\csname sseq@clearinfty\sseq@thename(9,3)\endcsname


\conditionally@traceoff
\foreach \x in {9,13,...,25}{
    %\d["hippo",/tikz/every to/.prefix code={\tracingall},/tikz/every to/.append code={\show\hi}]2 (\x,0)
    \d[opacity=0.5,use context,"\textup{hippo \x}"{sloped, above}]2 (\x,0)
    \d[draw=none]2 (\x,1)
    \draw (\x,1) -- (\x,2);
%
%
    %%
    \foreach \y in {7,15}{
        \d2 (\x,\y)
        \d2 (\x,\y+1,,2)
        \d2 (\x,\y+2,1,)
    }
}




\foreach \x in {4,8,...,24}{
    \d2 (\x,1)
    \d2 (\x,2)
    %s\ifnum\x<12\relax \draw (\x,\x)--(\x+2,\x+2); \fi
    \foreach \y in {7,15}{
        %\ifnum\y=15\relax \draw (\x,\y)--(\x+1,\y+2);\fi
         \d2 (\x,\y+1,,2)
        \d2 (\x,\y+2,1,)
        \d2 (\x,\y+3)
    }
}


\foreach \x in {4,8,..., 24,28}
    \foreach \y in {7,15}{
        \d3 (\x+2,\y,,1)
        \d3 (\x,\y+2,2,)
}

\foreach \x in {11,19,27}{
    \d4(\x,0)
}

\foreach \x in {12,20,28}
    \foreach \y in {7,15}{
        \d5(\x,\y)
}

\foreach \x in {10,18,26,34}{
    \d7(\x,1)
    \d7(\x-1,2)
    \d7(\x-2,3,,2)

    \foreach \y in {9,17}{
        \d7(\x,\y,1)
        \d7(\x-1,\y+1)
        \d7(\x-2,\y+2,,2)
    }
}

\d9(15,0)
\d9(14,1,,2)
\d10(13,2)

\d8(23,0)
\d8(22,1)
\foreach \x in {23,31} {
    \d8(\x-2,2,,2)
    \d9(\x-3,3)
}
\foreach \x in {16,24,32} {
    \d9(\x,7)
    \d8(\x-2,9,1,)
    \d8(\x-3,10,,2)
    \d9(\x-4,11)
}

\sseq@replace[fill=none,"\mathbb Z"](10,7)

\draw[<-,green] (2,3)--(0,5) -- (3,5);


\draw[<-,green,xshift=2,yshift=8] (2,3)--(0,5) -- (3,5);

%\showthe\sseq@x

\node at (6,5) {\textup{This will show up in all copies}};


%\d[dashed,"hi"] 8 (29,2,,2)
%\d13 (20,3)
\end{sseqdata}
%\show\sseq@savedpaths


\begin{sseqdata}[yscale=0.8]{ss2}
\cyclestyles{draw,circle,fill,inner sep=0.3ex}
            {draw,rectangle,fill,inner sep=0.3ex}
\degree{-1}{#1-1}

\xrange{1}{10}
\yrange{0}{10}

\place[needs tikz](1,0)

\foreach \x in {1,3,...,35} {\place(\x,0)}

\row{1}
\row{2}
\row{3}
\d[needs tikz,"\textup{a l a b e l}" {sloped,above}]  4   (9,0)
\foreach \x in {2,3,5,6}{
    \d["hi"]2 (\x,1)
}

\d3(7,0)

\end{sseqdata}


\begin{sseqpage}
\name{mysseq}
\page{4}
% \tracingall
\sseq@classoptions(11,0){red,fill=white}
\doptions4(19,0){dashed}
\doptions4(11,0){red}

\draw[red, dashed, ->, bend left=30, "\text{this doesn't go here}" {sloped,below}] (7,0) to (3,3);
\draw[blue,standard parse, "\text{this doesn't either}" sloped] (7,0) to (1,6);
\end{sseqpage}
\end{document}
\end{document}
\newpage

\printpage[name=mysseq, page=0,x range={1}{25},y range={0}{19}]
\newpage



%\pgfkeys{/tikz/sseqpages/global/mysseqoptions/.show code}

\newpage
\printpage[name=mysseq, page=2]
\newpage
\printpage[name=mysseq, page=3]

\newpage
\printpage[name=mysseq, page=4]
\newpage
\printpage[name=mysseq, page=5]
\newpage
\printpage[name=mysseq, page=7]
\newpage
\printpage[name=mysseq, page=8]
\newpage
\printpage[name=mysseq, page=9]
\newpage
\printpage[name=mysseq, page=10]
\end{document}




\begin{sseqpage}[page=0,label step=2,name=mysseq,scale=0.7]
\draw["\textup{this is it}" {above,sloped},dotted,->] (2,4) to (3,3);
\doptions2(5,0){red}
\doptions2(5,0){dashed}
\doptions2(9,0){red}
\doptions2(13,0){red}
\doptions2(17,0){red}
\doptions2(21,0){red}
\doptions2(25,0){red}

\doptions4(11,0){green}
\doptions4(19,0){green}
\doptions4(27,0){green}

\doptions9(16,7){blue}
\doptions9(24,7){blue}
\doptions7(10,9,1){blue}
\doptions7(18,9,1){blue}
\end{sseqpage}

\begin{sseqpage}[page=0,label step=2,name=mysseq,differentials=black]
\xrange{1}{25}
\yrange{0}{19}
\draw["\textup{this is it}" {above,sloped},dotted,->] (2,4) to (3,3);
%\tracingall
\doptions2(5,0){red}
%\error
\doptions2(9,0){red}
\doptions2(13,0){red}
\doptions2(17,0){red}
\doptions2(21,0){red}
\doptions2(25,0){red}

\doptions4(11,0){green}
\doptions4(19,0){green}
\doptions4(27,0){green}

\doptions9(16,7){blue}
\doptions9(24,7){blue}
\doptions7(10,9,1){blue}
\doptions7(18,9,1){blue}
\end{sseqpage}

\end{document}

\newpage

\begin{sseqpage}
\name{mysseq}
\page{4}
\draw[red, dashed, ->, "\text{this doesn't go here}" {sloped,below}] (7,0) to (3,3);
\end{sseqpage}
\newpage




\begin{sseqpage}
\name{ss2}
\page{0}
\doptions2(5,0){"hi!" {sloped, above}}
\doptions4(9,0){blue,dashed}
\end{sseqpage}


\vskip20pt

\begin{sseqpage}
\name{ss2}
\page{0}
\yrange{0}{5}
\end{sseqpage}

%\end{document}

\newpage




\begin{sseqpage}
\name{mysseq}
\page{0}
%\doptions 4 (9,0){dashed}
\end{sseqpage}




\begin{sseqpage}
\name{mysseq}
\page{0}
\xrange{1}{15}
\yrange{0}{15}
\doptions2(5,0){"x^3",dashed}
%\doptions2(9,0){blue}
\end{sseqpage}



\begin{sseqpage}
\name{ss2}
\page{0}
\doptions2(5,0){"hi!" {sloped, above}}
\doptions4(9,0){blue,dashed}
\end{sseqpage}

\vskip20pt

\begin{sseqpage}
\name{ss2}
\page{0}
\yrange{0}{5}
\end{sseqpage}






\pgfkeys{% `quotes' library support,
      /handlers/first char syntax/the character "/.initial=\sseq@forward@quotes,%
      /tikz/edge quotes mean={%
        edge node={node [execute at begin node=$,%$
                         execute at end node=$,%$
                         auto=right,every label,##2]{##1}}}}
\let\sseq@transform\sseq@errortransform % Disallow most coordinate transforms
\let\sseq@shifttransform\sseq@checkshifttransform % Allow shifts as long as they are by integers and have no units.
\newwrite\tempfile
\newtoks\temptoks
\@xp\expandallonce\csname sseq@savednodes@mysseq2\endcsname
\temptoks\@xp{\test}
\immediate\openout\tempfile=stuff.txt
\immediate\write\tempfile{\the\temptoks}
\immediate\closeout\tempfile 